\documentclass{article}
\usepackage{amsmath}   % para \frac, \dfrac, ecuaciones alineadas, etc.
\usepackage{amssymb}   % símbolos matemáticos adicionales
\usepackage{amsfonts}  % fuentes matemáticas adicionales
\usepackage{mathtools} % extensiones de amsmath, mejora el espaciado
\usepackage{enumitem}  % para listas con etiquetas personalizadas
\usepackage[margin=2.5cm]{geometry} % márgenes cómodos para ver mejor el bloque
\usepackage{lmodern}   % fuente moderna compatible con símbolos matemáticos
\usepackage{siunitx}   % opcional, útil para valores numéricos y unidades
\usepackage{hyperref} % para enlaces en el documento
\usepackage{graphicx}  % para incluir imágenes
\author{Francisco Ulloa}
\date{\today}
\begin{document}
\begin{center}
    {\Large \textbf{Escuela Politécnica Nacional}}\\[1em]
    {\large \textbf{Asignatura:} Métodos Numéricos}\\[0.5em]
    {\large \textbf{Nombre:} Francisco Ulloa}\\[0.5em]
    {\large \textbf{Fecha:} 5 de noviembre de 2025}\\[2em]
    \href{https://github.com/Fu5CHAR/Metodos_numericos_2025B_Ulloa-Francisco/tree/main}{Repositorio de GitHub}
\end{center}
\textbf{Curso de python principiante}\\
\begin{figure}[h!]
    \centering
    \includegraphics[width=0.6\textwidth]{Captura_curso_principiante.png}\\
    \caption{Curso de Python para principiantes.}
    \label{fig:Captura_curso_principiante}
\end{figure}
\textbf{Curso de python principiante}\\
\begin{figure}[h!]
    \centering
    \includegraphics[width=0.6\textwidth]{Captura_avanzado.png}\\
    \caption{Curso de Python avanzado.}
    \label{fig:Captura_avanzado}
\end{figure}



\end{document}