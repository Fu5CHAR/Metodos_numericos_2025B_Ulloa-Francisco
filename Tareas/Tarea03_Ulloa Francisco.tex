\documentclass{article}
\usepackage{amsmath}   % para \frac, \dfrac, ecuaciones alineadas, etc.
\usepackage{amssymb}   % símbolos matemáticos adicionales
\usepackage{amsfonts}  % fuentes matemáticas adicionales
\usepackage{mathtools} % extensiones de amsmath, mejora el espaciado
\usepackage{enumitem}  % para listas con etiquetas personalizadas
\usepackage[margin=2.5cm]{geometry} % márgenes cómodos para ver mejor el bloque
\usepackage{lmodern}   % fuente moderna compatible con símbolos matemáticos
\usepackage{siunitx}   % opcional, útil para valores numéricos y unidades
\usepackage{hyperref} % para enlaces en el documento
\author{Francisco Ulloa}
\date{\today}
\begin{document}
\begin{center}
    {\Large \textbf{Escuela Politécnica Nacional}}\\[1em]
    {\large \textbf{Asignatura:} Métodos Numéricos}\\[0.5em]
    {\large \textbf{Nombre:} Francisco Ulloa}\\[0.5em]
    {\large \textbf{Fecha:} 5 de noviembre de 2025}\\[2em]
    \href{https://github.com/Fu5CHAR/Metodos_numericos_2025B_Ulloa-Francisco/tree/main}{Repositorio de GitHub}
\end{center}

\textbf{1. Utilice aritmética de corte de tres dígitos para calcular las siguientes sumas.
Para cada parte, ¿qué método es más preciso y por qué?}

\textbf{a.  $\displaystyle\sum_{i=1}^{10} (\frac{1}{i^2})$ primero por $\displaystyle\frac{1}{1} + \frac{1}{4} + ... +\frac{1}{100}$ y luego por $\displaystyle\frac{1}{100} + \frac{1}{81} + ... + \frac{1}{1}$}\\
\\
\\
\bigskip $\displaystyle\sum_{i=1}^{10}\frac{1}{i^2}$ \\

Usaremos corte a 3 cifras significativas en cada suma parcial. \\

\textbf{Orden directo (i=1..10):} \\
Términos: \\
$\frac{1}{1^2}=1,\;\frac{1}{2^2}=0.25,\;\frac{1}{3^2}=0.111111\ldots,\;\frac{1}{4^2}=0.0625,\;\frac{1}{5^2}=0.04,$ \\
$\frac{1}{6^2}=0.027777\ldots,\;\frac{1}{7^2}=0.020408\ldots,\;\frac{1}{8^2}=0.015625,\;\frac{1}{9^2}=0.0123456\ldots,\;\frac{1}{10^2}=0.01$\\

Suma parcial (con corte a 3 cifras significativas después de cada adición):\\
$S_1 = 1$\\
$S_2 = \operatorname{chop}_3(1 + 0.25) = 1.25$\\
$S_3 = \operatorname{chop}_3(1.25 + 0.111111\ldots) = 1.36$\\
$S_4 = \operatorname{chop}_3(1.36 + 0.0625) = 1.42$\\
$S_5 = \operatorname{chop}_3(1.42 + 0.04) = 1.46$\\
$S_6 = \operatorname{chop}_3(1.46 + 0.027777\ldots) = 1.48$\\
$S_7 = \operatorname{chop}_3(1.48 + 0.020408\ldots) = 1.50$\\
$S_8 = \operatorname{chop}_3(1.50 + 0.015625) = 1.51$\\
$S_9 = \operatorname{chop}_3(1.51 + 0.0123456\ldots) = 1.52$\\
$S_{10} = \operatorname{chop}_3(1.52 + 0.01) = \boxed{1.53}$ \\

\smallskip
\textbf{Orden inverso (i=10..1):} \\
Términos en orden inverso: $0.01,\;0.0123456\ldots,\;0.015625,\;\dots,\;0.25,\;1$\\

Suma parcial (corte 3 cifras después de cada adición):\\
$S_1' = 0.01$\\
$S_2' = \operatorname{chop}_3(0.01 + 0.0123456\ldots) = 0.022$\\
$S_3' = \operatorname{chop}_3(0.022 + 0.015625) = 0.037$\\
$S_4' = \operatorname{chop}_3(0.037 + 0.020408\ldots) = 0.057$\\
$S_5' = \operatorname{chop}_3(0.057 + 0.027777\ldots) = 0.084$\\
$S_6' = \operatorname{chop}_3(0.084 + 0.04) = 0.124$\\
$S_7' = \operatorname{chop}_3(0.124 + 0.0625) = 0.186$\\
$S_8' = \operatorname{chop}_3(0.186 + 0.111111\ldots) = 0.297$\\
$S_9' = \operatorname{chop}_3(0.297 + 0.25) = 0.547$\\
$S_{10}' = \operatorname{chop}_3(0.547 + 1) = \boxed{1.54}$ \\

\smallskip
Valores de referencia (alta precisión): \\
$\displaystyle \sum_{i=1}^{10}\frac{1}{i^2} = 1.54976773\ldots$ \\

Errores absolutos: \\
Orden directo: $|1.54976773 - 1.53| \approx 0.01977$ \\
Orden inverso: $|1.54976773 - 1.54| \approx 0.00977$ \\

\textbf{Conclusión a):} la \textbf{orden inverso} (sumar primero los términos más pequeños) es más preciso. 
Porque la aritmética de corte provoca pérdida de precisión cuando se suman números de tamaños muy distintos. Sumar de menor a mayor reduce esa pérdida acumulada.
\bigskip
\textbf{b. $\displaystyle\sum_{i=1}^{10} (\frac{1}{i^3})$ primero por $\displaystyle \frac{1}{1} + \frac{1}{8} + \frac{1}{27} + ... + \frac{1}{1000}$ y luego por $\displaystyle\frac{1}{1000} + \frac{1}{729} + ... + \frac{1}{1}$}\\
\\
\textbf{b. } $\displaystyle\sum_{i=1}^{10}\frac{1}{i^3}$ \\

\textbf{Orden directo (i=1..10):} términos: \\
$1,\; \frac{1}{2^3}=0.125,\; \frac{1}{3^3}=0.037037\ldots,\; \frac{1}{4^3}=0.015625,\; \dots,\; \frac{1}{10^3}=0.001$\\

Suma parcial (corte 3 cifras):\\
$S_1 = 1$\\
$S_2 = \operatorname{chop}_3(1 + 0.125) = 1.12$\\
$S_3 = \operatorname{chop}_3(1.12 + 0.037037\ldots) = 1.15$\\
$S_4 = \operatorname{chop}_3(1.15 + 0.015625) = 1.16$\\
$S_5 = \operatorname{chop}_3(1.16 + 0.008) = 1.16$\\
$S_6 = \operatorname{chop}_3(1.16 + 0.0046296\ldots) = 1.16$\\
$S_7 = \operatorname{chop}_3(1.16 + 0.00291545\ldots) = 1.16$\\
$S_8 = \operatorname{chop}_3(1.16 + 0.001953125) = 1.16$\\
$S_9 = \operatorname{chop}_3(1.16 + 0.0013717\ldots) = 1.16$\\
$S_{10} = \operatorname{chop}_3(1.16 + 0.001) = \boxed{1.16}$ \\

\smallskip
\textbf{Orden inverso (i=10..1):} términos pequeños primero \\
Suma parcial (corte 3 cifras):\\
$S_1' = 0.001$\\
$S_2' = \operatorname{chop}_3(0.001 + 0.0013717\ldots) = 0.002$\\
$S_3' = \operatorname{chop}_3(0.002 + 0.001953125) = 0.003$\\
$S_4' = \operatorname{chop}_3(0.003 + 0.00291545\ldots) = 0.005$\\
$S_5' = \operatorname{chop}_3(0.005 + 0.0046296\ldots) = 0.009$\\
$S_6' = \operatorname{chop}_3(0.009 + 0.008) = 0.017$\\
$S_7' = \operatorname{chop}_3(0.017 + 0.015625) = 0.032$\\
$S_8' = \operatorname{chop}_3(0.032 + 0.037037\ldots) = 0.069$\\
$S_9' = \operatorname{chop}_3(0.069 + 0.125) = 0.194$\\
$S_{10}' = \operatorname{chop}_3(0.194 + 1) = \boxed{1.19}$ \\

\smallskip
Valor real: \\
$\displaystyle \sum_{i=1}^{10}\frac{1}{i^3} = 1.19753199\ldots$ \\

Errores absolutos: \\
Orden directo: $|1.19753199 - 1.16| \approx 0.03753$ \\
Orden inverso: $|1.19753199 - 1.19| \approx 0.00753$ \\

\textbf{Conclusión b):} nuevamente \textbf{sumar de menor a mayor} (orden inverso) es más preciso por la misma razón: minimiza la pérdida de información del término pequeño al sumarlo a acumulados grandes cuando se aplica corte (truncamiento).
\\
\textbf{2. La serie de Maclaurin para la función arcotangenete converge para $\displaystyle -1 < x <= 1 $ y está dada por
$\displaystyle \arctan(x) = \lim_{n\to\infty} P_n(x) = \lim_{n\to\infty}\sum_{i=1}^{n} (-1)^{i+1} \frac{x^{2i-1}}{2i-1}$}\\
\\
\\
\textbf{a. Utilice el hecho de que $\tan(\frac{\pi}{4}=1)$ para determinar el número n de términos de la serie que se necesita
sumar para garantizar  que $\displaystyle |4P_n(1) - \pi| < 10^{-3}$}\\
\\
\\
% --------------------------
% Cálculos: número de términos necesarios para arctan(1)
% --------------------------
\textbf{Serie:} \\
La serie de Maclaurin para $\arctan(x)$ es
$$
\arctan(x)=\sum_{i=1}^{\infty} (-1)^{i+1}\frac{x^{2i-1}}{2i-1}.
$$
Para $x=1$ obtenemos
$$
\arctan(1)=\sum_{i=1}^{\infty} (-1)^{i+1}\frac{1}{2i-1}=\frac{\pi}{4}.
$$

\bigskip
\textbf{Resto por criterio de serie alternante:} \\
Si $P_n(1)=\sum_{i=1}^{n} (-1)^{i+1}\frac{1}{2i-1}$ es el polinomio truncado de orden $n$, entonces el error de truncamiento satisface
$$
\left|\arctan(1)-P_n(1)\right|\le \frac{1}{2n+1},
$$
porque el siguiente término (en magnitud) es $\dfrac{1}{2n+1}$ y la serie es alternante con términos decrecientes en magnitud.

Multiplicando por $4$ (ya que usamos $4P_n(1)$ para aproximar $\pi$) obtenemos la cota:
$$
\left| \pi - 4P_n(1)\right| \le \frac{4}{2n+1}.
$$

\bigskip
\textbf{a) Condición }$\displaystyle \left|4P_n(1)-\pi\right|<10^{-3}$: \\

Usamos la cota anterior y exigimos
$$
\frac{4}{2n+1} < 10^{-3}.
$$

Resolviendo:
$$
2n+1 > \frac{4}{10^{-3}} = 4000 \quad\Longrightarrow\quad n > \frac{4000-1}{2} = 1999.5.
$$

Por lo tanto el menor entero $n$ que cumple la desigualdad es
$$
\boxed{n = 2000.}
$$

\bigskip

\textbf{b. El elnguaje de programación C++ requiere que el valor de $\pi$ se encuentre dentro de  $10^{-10}$.
¿Cuántos términos de la serie se necessitan sumar para obtener este grado de presición?}\\
\textbf{b) Condición }$\displaystyle \left|4P_n(1)-\pi\right|<10^{-10}$: \\
Usamos la misma cota:
$$
\frac{4}{2n+1} < 10^{-10}.
$$

Resolviendo:
$$
2n+1 > \frac{4}{10^{-10}} = 4\times 10^{10} \quad\Longrightarrow\quad n > \frac{4\cdot 10^{10}-1}{2}.
$$

Por lo tanto el menor entero $n$ que satisface la condición es
$$
\boxed{n = 20\,000\,000\,000.}
$$

\textbf{Observación:} sumar $20\,000\,000\,000$ términos es prácticamente inviable en una implementación directa (tiempo y memoria), pero la fórmula anterior nos da el número teórico de términos requeridos por el criterio alternante.
\\
\textbf{Pseudocódigo (usar la cota alternante para calcular $n$):} \\
\begin{verbatim}
Entrada: tol  # tolerancia deseada para |4 P_n(1) - pi|
Salida: n_min # número mínimo de términos necesarios

# Usando la cota |pi - 4 P_n(1)| <= 4/(2n+1)
1. Calcular N_real = (4 / tol - 1) / 2
2. n_min = ceil(N_real)
3. Retornar n_min
\end{verbatim}

\bigskip

\textbf{Pseudocódigo (si además se desea sumar realmente hasta n):} \\

\begin{verbatim}
Entrada: n  # número de términos a sumar
Salida: Pn = sum_{i=1}^n (-1)^{i+1} / (2i-1)

1. Pn = 0
2. for i = 1 to n do
3.     term = ((-1)^(i+1)) / (2*i - 1)
4.     Pn = Pn + term
5. end for
6. Retornar Pn
\end{verbatim}
\textbf{3. Otra fórmula para calcular $\pi$ se puede deducir a partir de la identidad $\displaystyle \frac{\pi}{4} = 4\arctan\frac{1}{5} - \arctan\frac{1}{239}$.
Determine el número de términos que se deben sumar para garantizar una aproximación de $\pi$ dentro de $10^{-3}$.}\\
\\
\textbf{4. Compare los siguientes tres algoritmos. ¿Cuándo es correcto el algoritmo de la parte 1a?}\\
\textbf{a.}\\
\\ENTRADA $n,x_1,x_2,...,x_n$.\\
\\SALIDA PRODUCT.\\
\textit{Paso 1} Determine PRODUCT = 0\\
\textit{Paso 2} Para $i=1,2,...,n$ haga\\
\begin{center}
    Determine PRODUCT = PRODUCT $ * x_i$\\
\end{center}
\textit{Paso 3} SALIDA PRODUCT\\
\begin{center}
    PARE
\end{center}
\textbf{b.}\\
ENTRADA $n,x_1,x_2,...,x_n$.\\
\\SALIDA PRODUCT.\\
\textit{Paso 1} Determine PRODUCT = 1\\
\textit{Paso 2} Para $i=1,2,...,n$ haga\\
\begin{center}
    Set PRODUCT = PRODUCT $* x_i$\\
\end{center}
\textit{Paso 3} SALIDA PRODUCT\\
\begin{center}
    PARE    
\end{center}
\textbf{c.}\\
ENTRADA $n,x_1,x_2,...,x_n$.\\
\\SALIDA PRODUCT.\\
\textit{Paso 1} Determine PRODUCT = 1\\
\textit{Paso 2} Para $i=1,2,...,n$ haga\\
\begin{center}
si $x_i = 0$ entonces determine PRODUCT = 0
    \\SALIDA PRODUCT;
    \\PARE\\
Determine PRODUCT = PRODUCT $* x_i$\\
\end{center}
\textit{Paso 3} SALIDA PRODUCT\\
\begin{center}
    PARE
\end{center}
Al comparar los tres algoritmos tenemos que:\\
-El algoritmo \textit{a} en todos los casos da como resultado 0 sin importar los valores de $x_i$.\\
-El segundo y tercer algoritmo obtinenen resultados iguales, pero este último mejora al algoritmo \textit{b} porque si alguno de los $x_i$ es 0, el producto total será 0 y no es necesario seguir multiplicando.\\
-Finalmente, el algoritmo \textit{a} dará un resultado correcto solo y solo si todos los valores de $x_i$ son iguales a 0 o si se espera obtener un 0 por respuesta.\\
\\
\textbf{5. a. ¿Cuántas multiplicaciones y sumas se requieren para determinar una suma de la forma 
\begin{center}
    $\sum_{i=1}^{n}*\sum_{j=1}^{i}a_ib_j$?
\end{center}}
\textbf{Ejercicio 5.}\\
\textbf{a) Cálculo del número de operaciones} \\
\[
S=\sum_{i=1}^{n}\Big(\sum_{j=1}^{i} a_i b_j\Big)
\]

Para cada $i$ fijo hay $i$ multiplicaciones y $(i-1)$ sumas. Al final se suman los $n$ resultados parciales:

\[
\text{Multiplicaciones}=\sum_{i=1}^{n} i = \frac{n(n+1)}{2},
\]
\[
\text{Sumas}=\sum_{i=1}^{n} (i-1) + (n-1)=\frac{n^2 + n - 2}{2}.
\]

\bigskip
\textbf{b. Modifique la suma en la parte a. un formato equivalente que reduzca el número de cálculos}\\
\textbf{b) Forma equivalente que reduce cálculos} \\
\[
S=\sum_{i=1}^{n}\Big(a_i\sum_{j=1}^{i} b_j\Big).
\]

Definiendo las sumas prefijo $B_i=\sum_{j=1}^{i} b_j$, se obtiene:
\[
S=\sum_{i=1}^{n} a_i B_i.
\]

\[
\text{Multiplicaciones}=n,\qquad \text{Sumas}=2n-2.
\]

\bigskip
Otra forma equivalente es:
\[
S=\sum_{j=1}^{n} b_j\Big(\sum_{i=j}^{n} a_i\Big).
\]
\\
\textbf{6. Escriba un algoritmo para sumar la serie finita $\displaystyle\sum_{i=1}^{n}x_i$ en orden inverso}\\
\\
\textbf{Algoritmo 6: Suma de una serie en orden inverso}\\

\textit{Entrada:} $n, x_1, x_2, ..., x_n$\\
\textit{Salida:} SUMA\\

\textit{Paso 1:} Determine SUMA = 0\\
\textit{Paso 2:} Para $i = n, n-1, ..., 1$ haga\\
\begin{center}
    SUMA = SUMA + $x_i$\\
\end{center}
\textit{Paso 3:} SALIDA SUMA\\
\textit{Paso 4:} PARE
\\
\textbf{7. Las ecuaciones 1.2 y 1.3 en la sección 1.2 proporcionan formas alternativas para las raíces $x_1$ y $x_2$ de
$ax^2 + bx + c = 0$. Construya un algoritmo con entrada a,b,c y salida $x_1,x_2$ que calcule las raíces $x_1,x_2$
(que pueden ser iguales con conjugados complejos) mediante fórmula para cada raíz}\\
\\
\textbf{Algoritmo 7: Determinación de las raíces de un polinomio cuadrático}\\

\textit{Entrada:} $a, b, c$\\
\textit{Salida:} $x_1, x_2$\\

\textit{Paso 1:} Calcular el discriminante $D = b^2 - 4ac$\\

\textit{Paso 2:} Si $D \ge 0$ entonces\\
\begin{enumerate}[label=\alph*)]
    \item Si $b \ge 0$ entonces
    \begin{center}
        $x_1 = \dfrac{2c}{-b - \sqrt{D}}$ \hspace{1cm} \textit{(raíz más pequeña, fórmula (1.3))}\\
        $x_2 = \dfrac{-b - \sqrt{D}}{2a}$ \hspace{1cm} \textit{(raíz más grande, fórmula (1.2))}
    \end{center}
    \item Sino
    \begin{center}
        $x_1 = \dfrac{-b + \sqrt{D}}{2a}$ \hspace{1cm} \textit{(raíz más grande, fórmula (1.2))}\\
        $x_2 = \dfrac{2c}{-b + \sqrt{D}}$ \hspace{1cm} \textit{(raíz más pequeña, fórmula (1.3))}
    \end{center}
\end{enumerate}

\textit{Paso 3:} Si $D < 0$ entonces\\
\begin{center}
    parte\_real = $\dfrac{-b}{2a}$\\
    parte\_imaginaria = $\dfrac{\sqrt{|D|}}{2a}$\\
    $x_1 =$ parte\_real $+$ parte\_imaginaria$i$\\
    $x_2 =$ parte\_real $-$ parte\_imaginaria$i$
\end{center}

\textit{Paso 4:} SALIDA $x_1, x_2$\\
\textit{Paso 5:} PARE
\\
\\

\textbf{8. Suponga que
\begin{center}
    $\displaystyle \frac{1-2x}{1-x+x^2} + \frac{2x-4x^3-8x^7}{1-x^4+x^8} + \frac{1+2x}{1+x+x^2}$.
\end{center}
para $x < 1 $ y si $ x = 0.25$. Escriba y ejecute un algoritmo que determine el número de términos necesarios en el lado izquierdo de la ecuación de tal forma que el lado izquierdo
difiera  del lado derecho en menos de $10^{-6}$.}\\
\\
\\
\textbf{Algoritmo 8: Determinación del número de términos necesarios en una serie}\\

\textit{Entrada:} $x$, Diferencia\\
\textit{Salida:} $n$\\

\textit{Paso 1:} Asignar $x = 0.25$, Diferencia = $10^{-6}$\\
\textit{Paso 2:} Calcular lado derecho: 
\[
\text{right\_side} = \frac{1 + 2x}{1 + x + x^2}
\]
\textit{Paso 3:} Inicializar suma\_izquierda = 0, $n = 0$\\

\textit{Paso 4:} Mientras sea verdadero, hacer:\\
\begin{enumerate}[label=\alph*)]
    \item Calcular numerador: 
    \[
    \text{numerador} = 2^n x^{2^n - 1} - 2^{n+1} x^{2^{n+1} - 1}
    \]
    \item Calcular denominador:
    \[
    \text{denominador} = 1 - x^{2^n} + x^{2^{n+1}}
    \]
    \item Calcular término:
    \[
    \text{término} = \frac{\text{numerador}}{\text{denominador}}
    \]
    \item Actualizar suma:
    \[
    \text{suma\_izquierda} = \text{suma\_izquierda} + \text{término}
    \]
    \item Si $|\text{suma\_izquierda} - \text{right\_side}| < \text{Diferencia}$, entonces
    \begin{center}
        SALIR del ciclo
    \end{center}
    \item Incrementar $n = n + 1$
\end{enumerate}

\textit{Paso 5:} Mostrar en pantalla:
\[
\text{``Se necesitan } n + 1 \text{ términos para que la suma difiera del lado derecho en menos de } 10^{-6}.'' 
\]
\textit{Paso 6:} PARE
\\

\end{document}